\documentclass[10pt]{article}
\usepackage[utf8]{inputenc}
\usepackage[T1]{fontenc}
\usepackage{amsmath}
\usepackage{amsfonts}
\usepackage{amssymb}
\usepackage[version=4]{mhchem}
\usepackage{stmaryrd}
\usepackage{bbold}

\begin{document}
Università degli Studi di Catania - Anno Accademico 2019/20

Corso di Laurea in Fisica

Prova scritta di Analisi Matematica 1

29 marzo 2021

\begin{enumerate}
  \item Studiare la funzione definita dalla legge
\end{enumerate}

\[
f(x)=e^{x} \sqrt{\frac{x+2}{x-3}}
\]

e tracciarne il grafico.

\begin{enumerate}
  \setcounter{enumi}{1}
  \item Studiare il carattere delle seguenti serie numeriche
\end{enumerate}

\[
\sum_{n=1}^{+\infty} \sqrt[3]{\sin ^{2} \frac{1}{(n+2)^{3}}}, \quad \sum_{n=2}^{+\infty} \frac{1}{(\log n)^{\log n}} .
\]

\begin{enumerate}
  \setcounter{enumi}{2}
  \item Sia \(X\) l'insieme piano definito dalle limitazioni
\end{enumerate}

\[
X=\left\{(x, y) \in \mathbb{R}^{2}: \quad x \geq 1, \quad 0 \leq y \leq \frac{1}{\sqrt{x}(x-2 \sqrt{x}+2)}\right\} .
\]

Stabilire se \(X\) é misurabile secondo Peano-Jordan e, in caso affermativo, calcolarne la misura.


\end{document}