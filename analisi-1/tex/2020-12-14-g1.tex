\documentclass[10pt]{article}
\usepackage[utf8]{inputenc}
\usepackage[T1]{fontenc}
\usepackage{amsmath}
\usepackage{amsfonts}
\usepackage{amssymb}
\usepackage[version=4]{mhchem}
\usepackage{stmaryrd}
\usepackage{bbold}

\begin{document}
Università degli Studi di Catania - Anno Accademico 2019/20

Corso di Laurea in Fisica

Prova scritta di Analisi Matematica 1 - gruppo 1

14 dicembre 2020

\begin{enumerate}
  \item Studiare la funzione definita dalla legge
\end{enumerate}

\[
f(x)=|x| \exp \left(\frac{x+1}{x-1}\right)
\]

e tracciarne il grafico.

\begin{enumerate}
  \setcounter{enumi}{1}
  \item Studiare il carattere della serie numerica
\end{enumerate}

\[
\sum_{n=1}^{+\infty} \frac{(-1)^{n}}{\sqrt{n+1}-n^{2}}
\]

\begin{enumerate}
  \setcounter{enumi}{2}
  \item Sia \(X\) l'insieme piano definito dalle limitazioni
\end{enumerate}

\[
X=\left\{(x, y) \in \mathbb{R}^{2} \quad: \quad \frac{\pi}{6} \leq x \leq \frac{3 \pi}{4}, \quad 0 \leq y \leq|\cos x| \log \frac{1+\sin ^{2} x}{\sin x}\right\}
\]

Stabilire se l'insieme \(X\) é misurabile secondo Peano-Jordan e, in caso affermativo, calcolarne la misura.


\end{document}