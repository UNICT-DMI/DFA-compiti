\documentclass[10pt]{article}
\usepackage[utf8]{inputenc}
\usepackage[T1]{fontenc}
\usepackage{amsmath}
\usepackage{amsfonts}
\usepackage{amssymb}
\usepackage[version=4]{mhchem}
\usepackage{stmaryrd}

\title{UNIVERSITÀ DEGLI STUDI DI CATANIA 
 Anno Accademico 2020 - 2021 
 Corso Di Laurea in FisiCA 
 Prova scritta di Analisi Matematica I del 25 gennaio 2021 
 gruppo 2 }

\author{}
\date{}


\begin{document}
\maketitle
\begin{enumerate}
  \item Studiare la funzione definita dalla legge
\end{enumerate}

\[
f(x)=\log \frac{x^{2}+1}{\left|x^{2}-x\right|}
\]

e tracciarne il grafico .

\begin{enumerate}
  \setcounter{enumi}{1}
  \item Calcolare l'insieme delle primitive della seguente funzione:
\end{enumerate}

\[
\frac{\cos x \operatorname{sen} 2 x}{\cos x+2}
\]

\begin{enumerate}
  \setcounter{enumi}{2}
  \item Studiare il carattere delle serie numeriche:
\end{enumerate}

\[
\quad \sum_{n=1}^{+\infty} \frac{2^{(n+1)^{2}}}{n !}, \quad \sum_{n=1}^{+\infty} \frac{1}{n^{2} \operatorname{sen} \frac{1}{\sqrt[3]{n}}}, \quad \sum_{n=2}^{+\infty}\left(\frac{1}{3}+\frac{2}{\log n}\right)^{n^{2}} .
\]


\end{document}