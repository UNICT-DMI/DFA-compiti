\documentclass[10pt]{article}
\usepackage[utf8]{inputenc}
\usepackage[T1]{fontenc}
\usepackage{amsmath}
\usepackage{amsfonts}
\usepackage{amssymb}
\usepackage[version=4]{mhchem}
\usepackage{stmaryrd}

\begin{document}
Università degli Studi di Catania - Anno Accademico 2019/20

Corso di Laurea in Fisica

Prova scritta di Analisi Matematica 1 - gruppo 3

13 luglio 2020

\begin{enumerate}
  \item Studiare la funzione reale di variabile reale definita dalla legge
\end{enumerate}

\[
f(x)=2 \log \frac{x-1}{x}+|x-2|
\]

e tracciarne il grafico.

\begin{enumerate}
  \setcounter{enumi}{1}
  \item Studiare, al variare del parametro reale \(x>0\), il carattere della seguente serie numerica:
\end{enumerate}

\[
\sum_{n=1}^{+\infty} \frac{n \sqrt{n}}{n^{2 x}+1}
\]

\begin{enumerate}
  \setcounter{enumi}{2}
  \item Calcolare, se esiste, il seguente integrale
\end{enumerate}

\[
\int_{1}^{+\infty} \frac{\arctan x}{(x+1)^{2}} d x
\]


\end{document}