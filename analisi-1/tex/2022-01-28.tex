\documentclass[10pt]{article}
\usepackage[utf8]{inputenc}
\usepackage[T1]{fontenc}
\usepackage{amsmath}
\usepackage{amsfonts}
\usepackage{amssymb}
\usepackage[version=4]{mhchem}
\usepackage{stmaryrd}
\usepackage{bbold}

\title{Corso di Laurea in Fisica }

\author{}
\date{}


\begin{document}
\maketitle
Anno Accademico 2021/22

Seconda verifica di Analisi Matematica 1

28 gennaio 2022

\section{TEORIA}
1 Enunciare e dimostrare il teorema sul limite delle funzioni monotone.

2 Sia \(\left\{a_{n}\right\}\) una successione numerica.

Dire, giustificando la risposta, se le seguenti affermazioni sono vere o false.

\[
\begin{aligned}
& \sup \left\{a_{n}\right\}=+\infty \quad \lim _{n \rightarrow+\infty} a_{n}=+\infty \\
& \lim _{n \rightarrow+\infty} a_{n}=+\infty \Longrightarrow \sup \left\{a_{n}\right\}=+\infty
\end{aligned}
\]

\section{EsERCIZI}
3 Determinare il dominio e la legge di definizione della funzione definita da

\[
f(x)=\lim _{n \rightarrow+\infty} \frac{\sqrt[n]{x^{2}+1}-1}{\log \left(1+\frac{|x|}{n}\right)}
\]

4 Calcolare, se esistono, i seguenti limiti

\[
\lim _{x \rightarrow 0} \frac{1}{x^{2}} \cos \frac{1}{x^{2}} \log \cos \left(x^{2}\right), \quad \lim _{x \rightarrow+\infty}\left(\frac{(1+x)^{1+x}}{x^{x}}-x^{2}\right) \sin ^{2} \frac{1}{x}
\]

5 Stabilire per quali valori del parametro reale \(k>0\) l'insieme numerico

risulta limitato .

\[
A=\left\{(-1)^{n} \frac{\log \left(1+\mathrm{e}^{n}\right)}{3 n^{k}+1}, \quad n \in \mathbb{N}\right\}
\]

6 Studiare il carattere della successione \(\left\{a_{n}\right\}\) definita da

\[
a_{1}=2, \quad a_{n+1}=\frac{n^{2}}{2 n-1} a_{n}, \quad \forall n \in \mathbb{N}
\]


\end{document}