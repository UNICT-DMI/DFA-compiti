\documentclass[10pt]{article}
\usepackage[utf8]{inputenc}
\usepackage[T1]{fontenc}
\usepackage{amsmath}
\usepackage{amsfonts}
\usepackage{amssymb}
\usepackage[version=4]{mhchem}
\usepackage{stmaryrd}
\usepackage{bbold}

\begin{document}
Anno Accademico 2021-2022

Corso di Laurea in Fisica

Prova scritta di Analisi Matematica 1

27 giugno 2022

1 Data la funzione definita dalla legge

\[
f(x)=\frac{x}{2}+\arctan \frac{|1-x|}{2 x-1}
\]

(i) determinarne il dominio e gli eventuali asintoti;

(ii) studiare la derivabilità, determinare gli eventuali punti di estremo relativo e gli intervalli in cui è monotona.

(iii) tracciare un grafico qualitativo di \(f\);

2 Stabilire se la funzione definita dalla legge

\[
f(x)=\frac{\log |x|}{(x+2)^{2}}
\]

è integrabile nell'intervallo \([-1,1]\) e, in caso affermativo, calcolare

\[
\int_{-1}^{1} f(x) \mathrm{d} x
\]

3 Studiare il carattere delle seguenti serie numeriche

\[
\sum_{n=1}^{+\infty} n^{2} \cos n^{2} \log \cos \frac{1}{n^{2}}, \quad \sum_{n=1}^{+\infty} \frac{3^{n}}{2^{n}+\log n} \sin \frac{2}{n !} .
\]

4 Determinare, al variare del parametro reale \(\lambda\) gli estremi dell'insieme numerico

\[
X_{\lambda}=\left\{(|x|+(\sin \lambda) x) \mathrm{e}^{-x^{2}}, \quad x \in \mathbb{R}\right\}
\]

5 Scrivere in forma algebrica le soluzioni complesse dell'equazione

\[
z^{2}|z|^{2}-(1+i) \bar{z}=0
\]

i) Durata: 3h min.

ii) Non si possono consultare libri o appunti.

iii) Gli studenti che hanno superato la prova intermedia devono svolgere solo gli esercizi 1, 2 e 3 .


\end{document}