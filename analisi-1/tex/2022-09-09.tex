\documentclass[10pt]{article}
\usepackage[utf8]{inputenc}
\usepackage[T1]{fontenc}
\usepackage{amsmath}
\usepackage{amsfonts}
\usepackage{amssymb}
\usepackage[version=4]{mhchem}
\usepackage{stmaryrd}

\begin{document}
Anno Accademico 2021-2022

Corso di Laurea in Fisica

Prova scritta di Analisi Matematica 1

9 settembre 2022

1 Data la funzione definita dalla legge

\[
f(x)=\frac{x-1}{x+3} \exp \left(-\frac{|x-1|}{x+3}\right)
\]

(i) determinarne il dominio e gli eventuali asintoti;

(ii) studiare la derivabilità, determinare gli eventuali punti di estremo relativo e gli intervalli in cui è monotona;

(iii) tracciare un grafico qualitativo di \(f\);

(iv) determinare l'immagine di \(f\).

2 Determinare \(F(x)\) primitiva di

\[
f(x)=\arctan \frac{|1-x|}{2 x-1}
\]

in \(] \frac{1}{2},+\infty[\) e tale che \(F(1)=0\).

3 Studiare, al variare del parametro reale \(\alpha\) il carattere della seguente serie numerica

\[
\sum_{n=1}^{+\infty}\left[\frac{(-2)^{n}}{(3 n) !}-n^{\alpha}\left(1-\cos \frac{1}{n}\right)\right] .
\]

4 Scrivere in forma algebrica le soluzioni complesse dell'equazione

\[
(z+i)^{2}=\frac{2-i}{2+i}
\]

i) Durata: 3h.

ii) Non si possono consultare libri o appunti.


\end{document}