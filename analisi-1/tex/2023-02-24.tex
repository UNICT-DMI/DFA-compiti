\documentclass[10pt]{article}
\usepackage[utf8]{inputenc}
\usepackage[T1]{fontenc}
\usepackage{amsmath}
\usepackage{amsfonts}
\usepackage{amssymb}
\usepackage[version=4]{mhchem}
\usepackage{stmaryrd}
\usepackage{bbold}

\title{Università degli Studi di Catania \\ Corso di Laurea in FISICA \\ Prova intermedia di Analisi Matematica 1 }

\author{}
\date{}


\begin{document}
\maketitle
24 febbraio 2020

1.Rispondere ad almeno una delle seguenti domande:

(1) Sia \(A \subseteq \mathbb{R}, A \neq \varnothing\).

Si dice che \(A\) è dotato di minimo se ... (completare la definizione).

Si dice che \(A\) è limitato inferiormente se ... (completare la definizione).

Se \(A\) è limitato inferiormente, si chiama estremo inferiore di \(A\)... (completare la definizione).

Stabilire se le seguenti affermazioni sono vere o false. Dimostrare quelle vere e portare un controesempio per quelle false.

Se \(A\) è dotato di minimo allora \(A\) è limitato inferiormente ;

se \(A\) è limitatato inferiormente allora \(A\) è dotato di minimo.

(2) Sia \(\left\{a_{n}\right\}\) una successione di numeri reali. Completare le seguenti definizioni.

Si dice che \(\left\{a_{n}\right\}\) è di Cauchy se

Si dice che \(\left\{a_{n}\right\}\) è limitata, se

Stabilire se le seguenti affermazioni sono vere o false. Per quelle false portare un controesempio.

Se \(\left\{a_{n}\right\}\) è di Cauchy allora \(\left\{a_{n}\right\}\) è limitata ;

se \(\left\{a_{n}\right\}\) è limitata allora \(\left\{a_{n}\right\}\) è di Cauchy.

\begin{enumerate}
  \setcounter{enumi}{1}
  \item Rispondere ad almeno una delle seguenti domande:
\end{enumerate}

(1) Enunciare e dimostrare il teorema sul limite delle funzioni monotone.

(2) Enunciare e dimostrare il teorema di Weierstrass. 3. Risolvere almeno uno dei seguenti esercizi:

(1) Studiare il carattere della successione

\[
\left\{\frac{\sqrt[n]{n !}}{n}\right\}
\]

(2) Calcolare, se esistono, i seguenti limiti

\[
\lim _{x \rightarrow+\infty} x^{4}\left(1+\frac{\sin ^{3} x}{x}\right), \lim _{x \rightarrow 0^{+}} \frac{\log \left(1+x+x^{2}\right)}{\sqrt{x}(1-\cos x)} .
\]

\begin{enumerate}
  \setcounter{enumi}{3}
  \item Risolvere almeno uno dei seguenti esercizi:
\end{enumerate}

(1) Sia \(\left\{a_{n}\right\}\) una successione numerica tale che

\[
a_{1}=1, \quad a_{n+1}=\left(n^{2}+1\right) a_{n} \quad \forall n>1 .
\]

Provare che

\[
a_{n}>0 \quad \forall n \in \mathbb{N} \text {. }
\]

Cosa si può dire del

\[
\lim _{n \rightarrow+\infty} \frac{1}{a_{n}} ?
\]

(2) Data la funzione definita dalla legge

\[
f(x)=\sup _{n \in \mathbb{N}}\left(\sqrt{\frac{x|x|}{x+2}}\right)^{n}
\]

\begin{enumerate}
  \item determinarne il dominio;

  \item stabilire se \(f\) è prolungabile per continuità in \(\mathbb{R}\) e, in caso affermativo, costruirne un prolungamento continuo.

\end{enumerate}

\end{document}