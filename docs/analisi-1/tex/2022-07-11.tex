\documentclass[10pt]{article}
\usepackage[utf8]{inputenc}
\usepackage[T1]{fontenc}
\usepackage{amsmath}
\usepackage{amsfonts}
\usepackage{amssymb}
\usepackage[version=4]{mhchem}
\usepackage{stmaryrd}

\begin{document}
Anno Accademico 2021-2022

Corso di Laurea in Fisica

Prova scritta di Analisi Matematica 1

11 luglio 2022

1 Data la funzione definita dalla legge

\[
f(x)=\exp \frac{1}{x \sqrt{|x|}-1}
\]

(i) determinarne il dominio e gli eventuali asintoti;

(ii) studiare la derivabilità, determinare gli eventuali punti di estremo relativo e gli intervalli in cui è monotona;

(iii) tracciare un grafico qualitativo di \(f\);

(iv) stabilire se \(f\) è invertibile nel suo insieme di definizione e, in caso affermativo, determinare il dominio e la legge di definizione della funzione inversa.

2 Calcolare, se esistono, i seguenti integrali definiti

\[
\int_{0}^{1} \frac{x+3}{x+\sqrt{x}} \mathrm{~d} x, \quad \int_{1}^{+\infty} \frac{x+3}{x+\sqrt{x}} \mathrm{~d} x
\]

3 Studiare il carattere delle seguenti serie numeriche

\[
\sum_{n=2}^{+\infty} \frac{(-1)^{n}}{n !} \log \left(1+\frac{(-1)^{n}}{n^{2}}\right), \quad \sum_{n=1}^{+\infty} \frac{n+3^{n}}{n 3^{n}-n+4} .
\]

4 Determinare per quali valori del parametro reale \(\lambda\) l'equazione

\[
\sqrt{x^{2}-3 x|x|}+\log \frac{1-x}{2+x}=\lambda
\]

ha soluzioni reali.

5 Scrivere in forma algebrica le radici quadrate del numero complesso

\[
w=\mathfrak{R e} \frac{2+4 i}{3+i}+i \mathfrak{I} \mathfrak{m}(2-i)^{3}
\]

i) Durata: 3h.

ii) Non si possono consultare libri o appunti.

iii) Gli studenti che hanno superato la prova intermedia devono svolgere solo gli esercizi 1,2 e 3 .


\end{document}