\documentclass[10pt]{article}
\usepackage[utf8]{inputenc}
\usepackage[T1]{fontenc}
\usepackage{amsmath}
\usepackage{amsfonts}
\usepackage{amssymb}
\usepackage[version=4]{mhchem}
\usepackage{stmaryrd}
\usepackage{bbold}

\title{Risolvere almeno uno dei seguenti esercizi: }

\author{}
\date{}


\begin{document}
\maketitle
Università degli Studi di Catania - Anno Accademico 2019/2020

Corso di Laurea in Fisica

Prova scritta di Analisi Matematica 2

9 settembre 2020

\begin{enumerate}
  \item Studiare la funzione definita dalla legge
\end{enumerate}

\[
f(x)=\sqrt{\frac{|x|-1}{x+2}}+|x|
\]

e tracciarne il grafico.

\begin{enumerate}
  \setcounter{enumi}{1}
  \item Calcolare l'area del seguente insieme piano
\end{enumerate}

\[
T=\left\{(x, y) \in \mathbb{R}^{2}: \quad-\frac{\pi}{2} \leq x \leq \frac{\pi}{6}, \quad 0 \leq y \leq \frac{|\sin x| \cos ^{2} x}{\cos x+2}\right\}
\]

\begin{enumerate}
  \setcounter{enumi}{2}
  \item Studiare il carattere delle seguenti serie numeriche
\end{enumerate}

\[
\sum_{n=1}^{+\infty} \frac{1}{n \sin \frac{1}{\sqrt[3]{n}}}, \quad \sum_{n=1}^{+\infty}(-1)^{n}\left(\frac{1}{2}+\frac{1}{\log n}\right)^{n}
\]


\end{document}