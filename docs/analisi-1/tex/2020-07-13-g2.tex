\documentclass[10pt]{article}
\usepackage[utf8]{inputenc}
\usepackage[T1]{fontenc}
\usepackage{amsmath}
\usepackage{amsfonts}
\usepackage{amssymb}
\usepackage[version=4]{mhchem}
\usepackage{stmaryrd}
\usepackage{bbold}

\begin{document}
Università degli Studi di Catania - Anno Accademico 2019/20

Corso di Laurea in Fisica

Prova scritta di Analisi Matematica 1 - gruppo 2

13 luglio 2020

\begin{enumerate}
  \item Studiare la funzione definita dalla legge
\end{enumerate}

\[
f(x)=|2 x| \exp \left(\frac{1}{2 x-1}\right)
\]

e tracciarne il grafico.

\begin{enumerate}
  \setcounter{enumi}{1}
  \item Studiare il carattere della serie numerica
\end{enumerate}

\[
\sum_{n=1}^{+\infty} \frac{\log n}{n}\left(1-\cos \frac{1}{n}\right)
\]

\begin{enumerate}
  \setcounter{enumi}{2}
  \item Calcolare, se esistono, tutte le primitive in \(\mathbb{R}\) della funzione \(f: \mathbb{R} \rightarrow \mathbb{R}\) definita dalla legge
\end{enumerate}

\[
f(x)= \begin{cases}1-x \mathrm{e}^{x^{2}} & \text { se } x \geq 0 \\ \frac{\log \left(1+x^{2}\right)}{x^{2}} & \text { se } x<0\end{cases}
\]


\end{document}