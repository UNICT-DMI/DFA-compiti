\documentclass[10pt]{article}
\usepackage[utf8]{inputenc}
\usepackage[T1]{fontenc}
\usepackage{amsmath}
\usepackage{amsfonts}
\usepackage{amssymb}
\usepackage[version=4]{mhchem}
\usepackage{stmaryrd}
\usepackage{bbold}

\title{TEORIA }

\author{}
\date{}


\begin{document}
\maketitle
Anno Accademico 2021/22

Corso di Laurea in Fisica

Prova intermedia di Analisi Matematica 1

4 marzo 2022

(1) Enunciare e dimostrare il teorema sull'esistenza delle radici \(n\)-esime di un numero complesso.

(2) Sia \(\left\{a_{n}\right\}\) una successione numerica. Dire giustificando le risposte quali relazioni intercorrono tra la convergenza e la limitatezza di \(\left\{a_{n}\right\}\).

\section{ESERCIZI}
1 Determinare il dominio delle funzioni definite dalle seguenti leggi

\[
\begin{gathered}
f(x)=\sup \left\{\left(x^{2 \log \frac{1}{x}}\right)^{n}, \quad n \in \mathbb{N}\right\} ; \\
g(x)=\sqrt{x^{2}-x|x|} \arctan \frac{x}{x-1} .
\end{gathered}
\]

2 Calcolare, se esistono, i seguenti limiti

\[
\lim _{x \rightarrow 0^{+}} \frac{2^{x}-2+\cos x+\sqrt{x}}{\sqrt[4]{x \sin x}} \lim _{x \rightarrow+\infty} \frac{\sqrt{x}-2 \arctan x^{3}}{x^{2}+3 x} \sin ^{2} x .
\]

3 Scrivere in forma algebrica le soluzioni complesse della seguente equazione

\[
z^{4} \bar{z}(1+i)=|z|^{2}
\]

4 Studiare il carattere della successione \(\left\{a_{n}\right\}\) definita da

\[
a_{1}=\frac{5}{2}, \quad a_{n+1}=\left(1+a_{n}\right)^{2} \quad \forall n \in \mathbb{N} .
\]


\end{document}