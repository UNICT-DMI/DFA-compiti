\documentclass[10pt]{article}
\usepackage[utf8]{inputenc}
\usepackage[T1]{fontenc}
\usepackage{amsmath}
\usepackage{amsfonts}
\usepackage{amssymb}
\usepackage[version=4]{mhchem}
\usepackage{stmaryrd}
\usepackage{bbold}

\begin{document}
Università degli Studi di Catania - Anno Accademico 2019/20

Corso di Laurea in Fisica

Prova scritta di Analisi Matematica 1 - gruppo 2

14 dicembre 2020

\begin{enumerate}
  \item Studiare la funzione definita dalla legge
\end{enumerate}

\[
f(x)=|2 x+1| \exp \left(\frac{x+1}{x}\right)
\]

e tracciarne il grafico.

\begin{enumerate}
  \setcounter{enumi}{1}
  \item Studiare il carattere della serie numerica
\end{enumerate}

\[
\sum_{n=1}^{+\infty}(-1)^{n} \log \left(n+\frac{1}{n^{2}}\right)
\]

\begin{enumerate}
  \setcounter{enumi}{2}
  \item Sia \(X\) l'insieme piano definito dalle limitazioni
\end{enumerate}

\[
X=\left\{(x, y) \in \mathbb{R}^{2}: \frac{-\pi}{6} \leq x \leq \frac{\pi}{4}, \quad 0 \leq y \leq(\sin 2 x) \arctan (\sin x+|\sin x|)\right\}
\]

Stabilire se \(X\) é misurabile secondo Peano-Jordan e, in caso affermativo, calcolarne la misura.


\end{document}