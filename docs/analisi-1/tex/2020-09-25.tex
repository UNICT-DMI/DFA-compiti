\documentclass[10pt]{article}
\usepackage[utf8]{inputenc}
\usepackage[T1]{fontenc}
\usepackage{amsmath}
\usepackage{amsfonts}
\usepackage{amssymb}
\usepackage[version=4]{mhchem}
\usepackage{stmaryrd}

\begin{document}
Università degli Studi di Catania - Anno Accademico 2019/20

Corso di Laurea in Fisica

Prova scritta di Analisi Matematica 1

25 settembre 2020

\begin{enumerate}
  \item Studiare la funzione definita dalla legge
\end{enumerate}

\[
f(x)=\exp (-x) \sqrt[3]{x+2}
\]

e tracciarne il grafico.

\begin{enumerate}
  \setcounter{enumi}{1}
  \item Studiare il carattere della serie numerica
\end{enumerate}

\[
\sum_{n=1}^{+\infty}\left[\frac{\arctan 2^{n}}{n^{2}}+\frac{2^{n}}{n^{\log n}}\right]
\]

\begin{enumerate}
  \setcounter{enumi}{2}
  \item Calcolare il seguente integrale indefinito
\end{enumerate}

\[
\int \frac{\cos x}{2 \sin x-\cos ^{2} x+6} \mathrm{~d} x
\]


\end{document}