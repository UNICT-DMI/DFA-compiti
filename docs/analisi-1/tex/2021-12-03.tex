\documentclass[10pt]{article}
\usepackage[utf8]{inputenc}
\usepackage[T1]{fontenc}
\usepackage{amsmath}
\usepackage{amsfonts}
\usepackage{amssymb}
\usepackage[version=4]{mhchem}
\usepackage{stmaryrd}
\usepackage{bbold}

\title{TEORIA }

\author{}
\date{}


\begin{document}
\maketitle
Anno Accademico 2021/22

Corso di Laurea in Fisica

Prima verifica di Analisi Matematica 1

3 dicembre 2021

1 i) Dare la definizione di maggiorante, minorante, massimo, minimo di un insieme.

ii) Dare la definizione di insieme limitato superiormente e non limitato inferiormente.

iii) Dimostrare il teorema di esistenza dell'estremo superiore.

2 Siano \(A \subseteq \mathbb{R}, A \neq \emptyset\) e \(f: A \rightarrow \mathbb{R}\)

i) Dare la definizione di funzione invertibile e di funzione strettamente monotona.

ii) Dire, giustificando la risposta se la seguente affermazione è vera o falsa

\[
f \text { è invertibile in } A \Longleftrightarrow f \text { è strettamente monotona in } A \text {. }
\]

ESERCIZI

3 Determinare il dominio delle funzioni definite dalle seguenti leggi

\[
f(x)=\arcsin \frac{1}{\log x}, \quad g(x)=\log \frac{\sqrt{x^{2}-2 x}+|x|}{x+1}
\]

4 Risolvere in \(\mathbb{C}\) l'equazione

\[
\bar{z} z^{2}+2(1+i)|z|^{2}+\sqrt{3} \bar{z}=0 .
\]

5 Determinare, al variare del parametro reale \(k>0\), l'estremo superiore e l'estremo inferiore dell'insieme numerico

\[
A=\left\{k^{\sqrt{n^{2}+2}-n}, \quad n \in \mathbb{N}\right\} .
\]


\end{document}