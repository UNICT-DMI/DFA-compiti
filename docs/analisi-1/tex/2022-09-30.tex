\documentclass[10pt]{article}
\usepackage[utf8]{inputenc}
\usepackage[T1]{fontenc}
\usepackage{amsmath}
\usepackage{amsfonts}
\usepackage{amssymb}
\usepackage[version=4]{mhchem}
\usepackage{stmaryrd}

\begin{document}
Anno Accademico 2021-2022

Corso di Laurea in Fisica

Prova scritta di Analisi Matematica 1

30 settembre 2022

1 Data la funzione definita dalla legge

\[
f(x)=x \sqrt{\frac{|x+1|}{x+2}}
\]

(i) determinarne il dominio e gli eventuali asintoti;

(ii) studiare la derivabilità, determinare gli eventuali punti di estremo relativo e gli intervalli in cui è monotona;

(iii) tracciare un grafico qualitativo di \(f\);

(iv) determinare l'immagine di \(f\).

2 Calcolare, se esiste, il seguente integrale definito

\[
\int_{0}^{1} \frac{1}{\sqrt{x}} \log \sqrt{4-x} \mathrm{~d} x
\]

3 Studiare, al variare del parametro reale \(\alpha>0\) il carattere della seguente serie numerica

\[
\sum_{n=1}^{+\infty} \frac{4 n^{\alpha}-1}{n^{3}+1} \tan \frac{1}{n^{2}} .
\]

4 Determinare il numero di soluzioni dell'equazione

\[
x^{3}+x=\arctan x
\]

5 Scrivere in forma algebrica le soluzioni complesse dell'equazione

\[
\bar{z}\left(i z^{2}+1\right)=(-1+i)|z|^{2}
\]

i) Durata: 3h.

ii) Non si possono consultare libri o appunti.


\end{document}