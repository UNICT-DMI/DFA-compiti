\documentclass[10pt]{article}
\usepackage[utf8]{inputenc}
\usepackage[T1]{fontenc}
\usepackage{amsmath}
\usepackage{amsfonts}
\usepackage{amssymb}
\usepackage[version=4]{mhchem}
\usepackage{stmaryrd}
\usepackage{bbold}

\begin{document}
Università degli Studi di Catania - Anno Accademico 2019/20

Corso di Laurea in Fisica

Prova scritta di Analisi Matematica 1

15 febbraio 2021

\begin{enumerate}
  \item Studiare la funzione definita dalla legge
\end{enumerate}

e tracciarne il grafico.

\[
f(x)=x+\log \frac{x^{2}-4}{3 x}
\]

\begin{enumerate}
  \setcounter{enumi}{1}
  \item Studiare il carattere delle seguenti serie numeriche
\end{enumerate}

\[
\sum_{n=1}^{+\infty} \frac{n+1}{n+2} \tan \frac{1}{\sqrt{n}}, \quad \sum_{n=1}^{+\infty} e^{-n}\left(1+\frac{3}{n}\right)^{n^{2}} .
\]

\begin{enumerate}
  \setcounter{enumi}{2}
  \item Sia \(X\) l'insieme piano definito dalle limitazioni
\end{enumerate}

\[
X=\left\{(x, y) \in \mathbb{R}^{2}: \quad-1 \leq x \leq 2, \quad 0 \leq y \leq e^{|x|} \frac{e^{2 x}+1}{2 e^{x}+1}\right\}
\]

Stabilire se \(X\) é misurabile secondo Peano-Jordan e, in caso affermativo, calcolarne la misura.


\end{document}