\documentclass[10pt]{article}
\usepackage[utf8]{inputenc}
\usepackage[T1]{fontenc}
\usepackage{amsmath}
\usepackage{amsfonts}
\usepackage{amssymb}
\usepackage[version=4]{mhchem}
\usepackage{stmaryrd}

\title{Università degli Studi di Catania - Anno Accademico 2019/20 
 Corso di Laurea in Fisica 
 Prova scritta di Analisi Matematica 1 
 8 ottobre 2020 }

\author{}
\date{}


\begin{document}
\maketitle
\begin{enumerate}
  \item Studiare la funzione definita dalla legge
\end{enumerate}

\[
f(x)=\arctan \frac{x}{|x|-1}
\]

e tracciarne il grafico.

\begin{enumerate}
  \setcounter{enumi}{1}
  \item Studiare, al variare del parametro reale \(x\), il carattere della serie numerica
\end{enumerate}

\[
\sum_{n=1}^{+\infty} \frac{n+2}{n^{x}+4} \sin \frac{1}{n^{2}}
\]

\begin{enumerate}
  \setcounter{enumi}{2}
  \item Stabilire se la funzione definita dalla legge
\end{enumerate}

\[
f(x)=\frac{1}{x^{4}} \cos ^{2} \frac{1}{x^{3}}
\]

é integrabile in senso improprio in \([1,+\infty[\). In caso affermativo, calcolare

\[
\int_{1}^{+\infty} f(x) \mathrm{d} x
\]


\end{document}