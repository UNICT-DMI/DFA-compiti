\documentclass[10pt]{article}
\usepackage[utf8]{inputenc}
\usepackage[T1]{fontenc}
\usepackage{amsmath}
\usepackage{amsfonts}
\usepackage{amssymb}
\usepackage[version=4]{mhchem}
\usepackage{stmaryrd}
\usepackage{bbold}

\begin{document}
Università degli Studi di Catania - Anno Accademico 2018/19

Corso di Laurea in Fisica

Prova scritta di Analisi Matematica 2

24 aprile 2019

\begin{enumerate}
  \item Dato il campo vettoriale
\end{enumerate}

\[
\mathbf{F}(x, y)=\left[\frac{\varphi(y)}{x}+\cos y\right] \mathbf{i}+[2 y \log x-x \sin y] \mathbf{j}
\]

determinare la funzione \(\varphi \in C^{1}(\mathbb{R})\) in modo che \(\varphi(0)=1\) e \(\mathbf{F}\) sia conservativo. Del campo cosí ottenuto determinare il potenziale che si annulla nel punto \(\left(\frac{\pi}{3}, 1\right)\).

\begin{enumerate}
  \setcounter{enumi}{1}
  \item Determinare l'integrale generale dell'equazione differenziale
\end{enumerate}

\[
y^{\prime \prime}+y^{\prime}-6 y=e^{k x}
\]

al variare del parametro reale \(k\).

\begin{enumerate}
  \setcounter{enumi}{2}
  \item Calcolare
\end{enumerate}

\[
\iiint_{D}\left(x^{2}+y^{2}\right) \mathrm{d} x \mathrm{~d} y \mathrm{~d} z
\]

essendo

\[
D=\left\{(x, y, z) \in \mathbb{R}^{3}: x^{2}+y^{2} \leq 1, \quad x^{2}+y^{2}+z \geq 1, \quad z \leq 3\right\} .
\]

\begin{enumerate}
  \setcounter{enumi}{3}
  \item Data la funzione definita dalla legge
\end{enumerate}

\[
f(x, y)=x^{2} \log (1+y)+x^{2} y^{2}
\]

i) determinare il dominio e gli eventuali estremi relativi;

ii) determinare, se esistono, gli eventuali estremi assoluti.


\end{document}