\documentclass[10pt]{article}
\usepackage[utf8]{inputenc}
\usepackage[T1]{fontenc}
\usepackage{amsmath}
\usepackage{amsfonts}
\usepackage{amssymb}
\usepackage[version=4]{mhchem}
\usepackage{stmaryrd}
\usepackage{bbold}

\title{Corso di Laurea in Fisica (L-30) }

\author{}
\date{}


\begin{document}
\maketitle
Anno Accademico 2021/22

Prova scritta di Analisi Matematica 2

14 febbraio 2022

1 Data la funzione definita dalla legge

\[
f(x, y)=\frac{x|x|}{\mathrm{e}^{x}-\mathrm{e}^{-y}}
\]

determinarne gli eventuali punti di estremo relativo. \(f\) é dotata di estremi assoluti? \(f\) é limitata? Giustificare le risposte.

2 Calcolare

\[
\iint_{T} \mathrm{e}^{2(x+y)} \mathrm{d} x \mathrm{~d} y
\]

essendo \(T\) il dominio piano limitato delimitato dalle rette di equazione \(y=x, y=x-2, y=-x\) e \(y=-x+2\).

3 Siano \(D=\left\{(u, v) \in \mathbb{R}^{2}: \quad 2 u \leq u^{2}+v^{2} \leq 1\right\}\) ed \(S\) la superficie di equazioni parametriche

\[
\left\{\begin{array}{l}
x=u+v \\
y=u-v \\
z=u^{3}
\end{array} \quad(u, v) \in D .\right.
\]

Calcolare il flusso del campo vettoriale

\[
\mathbf{F}=\left(x^{2}-y^{2}, 0,\left(x^{2}-y^{2}\right)(x+y)^{2}\right)
\]

attraverso la superficie \(S\) orientata con la normale verso l'alto.

4 Studiare la convergenza puntuale ed uniforme della serie

\[
\sum_{n=1}^{+\infty} \frac{n|x| \arctan \left(x^{3}\right)}{1+n^{3}|x|^{3}}
\]

Durata: 3 ore


\end{document}