\documentclass[10pt]{article}
\usepackage[utf8]{inputenc}
\usepackage[T1]{fontenc}
\usepackage{amsmath}
\usepackage{amsfonts}
\usepackage{amssymb}
\usepackage[version=4]{mhchem}
\usepackage{stmaryrd}
\usepackage{bbold}

\title{Università degli Studi di Catania 
 Corso di Laurea in Fisica 
 Prova scritta di Analisi Matematica 2 }

\author{}
\date{}


\begin{document}
\maketitle
16 dicembre 2019

(1) Calcolare il flusso del campo vettoriale

\[
\mathbf{F}=\left(x^{3}, y^{3}+z^{3}, z\right)
\]

uscente dal solido \(E \subseteq \mathbb{R}^{3}\) delimitato dal piano \(z=0\) e dalle superfici

\[
\begin{gathered}
\left\{(x, y, z) \in \mathbb{R}^{3}: z=x+2\right\} \\
\left\{(x, y, z) \in \mathbb{R}^{3}: x^{2}+y^{2}=1\right\}
\end{gathered}
\]

(2) Determinare per quali valori del parametro reale \(k\) il campo vettoriale

\[
\mathbf{F}=\left(-\frac{k}{1+y^{2}}+4 x, \frac{2 x y}{\left(1+y^{2}\right)^{2}}\right)
\]

è conservativo. Per tali valori di \(k\) calcolare il lavoro compito da \(\mathbf{F}\) lungo la curva \(\varphi(t)=(1+\cos t, \sin t), t \in[0, \pi]\) orientata nel verso delle \(t\) crescenti.

(3) Data la funzione definita dalla legge

\[
f(x, y)=x^{3}+3 x y^{2}-3\left(x^{2}+y^{2}\right)+4
\]

determinarne gli estremi assoluti, se esistono, nell'insieme

\[
X=\left\{(x, y) \in \mathbb{R}^{2}: \quad x^{2}+y^{2} \leq 4, \quad x \geq 0\right\}
\]

(4) Data la funzione definita dalla legge

\[
\begin{cases}\sin x \frac{e^{x y}-1}{\sqrt{x}^{2}+y^{2}} & \text { se }(x, y) \neq 0 \\ 0 & \text { se }(x, y)=0\end{cases}
\]

i) studiarne la continuità in \(\mathbb{R}^{2}\);

ii) calcolarne, se esistono, le derivate parziali prime nel punto \((0,0)\);

iii) studiarne la differenziabilità nel punto \((0,0)\).

(5) Calcolare

\[
\iiint_{D}(z+y) \mathrm{d} x \mathrm{~d} y \mathrm{~d} z
\]

essendo

\[
D=\left\{(x, y, z) \in \mathbb{R}^{3}: \quad x^{2}+y^{2} \leq(z-9)^{2}, \quad 0 \leq z \leq 3\right\} .
\]


\end{document}