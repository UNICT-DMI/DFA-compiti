\documentclass[10pt]{article}
\usepackage[utf8]{inputenc}
\usepackage[T1]{fontenc}
\usepackage{amsmath}
\usepackage{amsfonts}
\usepackage{amssymb}
\usepackage[version=4]{mhchem}
\usepackage{stmaryrd}
\usepackage{bbold}

\title{Corso di Laurea in Fisica (L-30) }

\author{}
\date{}


\begin{document}
\maketitle
Anno Accademico 2020/21

Prova scritta di Analisi Matematica 2

21 giugno 2021

1 Provare che la serie di funzioni

\[
\sum_{n=0}^{+\infty} e^{-n^{2}} \sin \left(n^{2} x\right)
\]

converge uniformemente in \(\mathbb{R}\). Detta \(f\) la sua funzione somma

i) provare che \(f\) è dotata di derivata seconda in \(\mathbb{R}\);

ii) provare che \(f\) è dotata di derivata di ogni ordine in \(\mathbb{R}\).

2 Studiare la continuità, l'esistenza delle derivate parziali prime e la differenziabilità nel punto \((0,0)\) della funzione reale di due variabili reali definita dalla legge

\[
f(x, y)=\left\{\begin{array}{lll}
\frac{2 y^{2}-x^{2}}{x^{2}+y^{2}}+2 y^{2}-x & \text { se } & (x, y) \neq(0,0) \\
-1 & \text { se } & (x, y)=(0,0) .
\end{array}\right.
\]

Determinare, se esistono, gli estremi assoluti nell'insieme

\[
X=\left\{(x, y) \in \mathbb{R}^{2}: 1 \leq x^{2}+y^{2} \leq 9, x \geq 0, y \geq 0\right\} .
\]

3 Calcolare il lavoro del campo vettoriale

\[
\mathbf{F}=\left(\frac{3 x}{x^{2}+y^{2}+1},-\frac{3 y}{x^{2}+y^{2}+1},\right)
\]

lungo la frontiera del dominio

\[
X=\left\{(x, y) \in \mathbb{R}^{2}: x^{2}+y^{2} \leq 1,|y| \leq x\right\}
\]

percorsa in senso antiorario. \(\mathbf{F}\) è conservativo?

4 Calcolare il flusso del campo vettoriale

\[
\mathbf{F}=\left(x^{2} y, y x^{2}, x y z\right)
\]

attraverso la frontiera del dominio

\[
T=\left\{(x, y, z) \in \mathbb{R}^{3}: \sqrt{x^{2}+y^{2}} \leq z \leq 2-\left(x^{2}+y^{2}\right)\right\} .
\]

5 Determinare la soluzione del problema di Cauchy

\[
\left\{\begin{array}{l}
x^{3} y^{\prime \prime \prime}-2 x^{2} y^{\prime \prime}+3 x y^{\prime}-3 y=x^{3} \log x \\
y(1)=1 \\
y^{\prime}(1)=y^{\prime \prime}(1)=0
\end{array}\right.
\]

Gli studenti che hanno superato la prova intermedia sono tenuti a svolgere solo gli esercizi 3,4 e 5.

Durata: 3 ore


\end{document}