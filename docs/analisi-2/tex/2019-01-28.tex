\documentclass[10pt]{article}
\usepackage[utf8]{inputenc}
\usepackage[T1]{fontenc}
\usepackage{amsmath}
\usepackage{amsfonts}
\usepackage{amssymb}
\usepackage[version=4]{mhchem}
\usepackage{stmaryrd}
\usepackage{bbold}

\begin{document}
Università degli Studi di Catania - Anno Accademico 2017/18

Corso di Laurea in Fisica

Prova scritta di Analisi Matematica 2

28 gennaio 2019

\begin{enumerate}
  \item Determinare gli eventuali punti di estremo relativo e assoluto della funzione definita dalla legge
\end{enumerate}

\[
f(x, y)=y^{2}\left(y^{2}+x^{2}-2 x\right)
\]

\begin{enumerate}
  \setcounter{enumi}{1}
  \item Calcolare
\end{enumerate}

\[
\int_{T}\left(x y^{2}+\frac{z^{2}}{x}\right) d x d y d z
\]

essendo

\(T=\left\{(x, y, z) \in \mathbb{R}^{3}: \quad x^{4} \leq x^{2} y^{2}+z^{2} \leq 4 x^{4}, \quad 0 \leq x y \leq z, \quad 1 \leq x \leq 2\right\}\)

\begin{enumerate}
  \setcounter{enumi}{2}
  \item Calcolare
\end{enumerate}

\[
\int_{\gamma}\left(\log y+\frac{y}{x^{2}+y^{2}}\right) d x+\left(\frac{x}{y}-\frac{x}{x^{2}+y^{2}}\right) d y
\]

essendo \(\gamma\) la curva di equazioni parametriche

\[
\left\{\begin{array}{l}
x(t)=t+\cos ^{3} t \\
y(t)=1+\sin ^{3} t
\end{array} t \in[0,2 \pi]\right.
\]

orientata nel verso delle \(t\) crescenti.

\begin{enumerate}
  \setcounter{enumi}{3}
  \item Studiare la convergenza puntuale ed uniforme della serie
\end{enumerate}

\[
\sum_{n=1}^{+\infty} \frac{\log n}{2^{n}+3^{n}}(x-1)^{n}
\]

\begin{enumerate}
  \setcounter{enumi}{4}
  \item Determinare il flusso del campo vettoriale
\end{enumerate}

\[
\mathbf{F}(x, y, z)=\left(x, z^{2}, y^{2} z\right)
\]

attraverso la superficie di equazione cartesiana

\[
z=\sqrt{x^{2}+y^{2}}, \quad 1 \leq x^{2}+y^{2} \leq 4
\]

e orientata con la normale verso l'alto


\end{document}