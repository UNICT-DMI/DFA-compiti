\documentclass[10pt]{article}
\usepackage[utf8]{inputenc}
\usepackage[T1]{fontenc}
\usepackage{amsmath}
\usepackage{amsfonts}
\usepackage{amssymb}
\usepackage[version=4]{mhchem}
\usepackage{stmaryrd}
\usepackage{bbold}

\begin{document}
Università degli Studi di Catania - Anno Accademico 2018/19

Corso di Laurea in Fisica

Prova scritta di Analisi Matematica 2

24 giugno 2019

\begin{enumerate}
  \item Stabilire se la forma differenziale
\end{enumerate}

\[
w(x, y)=\frac{2 x-y+1}{x^{2}+(y-1)^{2}} \mathrm{~d} x+\frac{x+2 y-2}{x^{2}+(y-1)^{2}} \mathrm{~d} y
\]

è esatta nel suo insieme di definizione. Calcolare poi

\[
\int_{\gamma} w
\]

essendo \(\gamma\) la curva di rappresentazione parametrica

\[
\left\{\begin{array}{l}
x=2 \cos t \\
y=\sin t-1
\end{array} \quad t \in[0,2 \pi]\right.
\]

percorsa nel verso delle \(t\) crescenti.

\begin{enumerate}
  \setcounter{enumi}{1}
  \item Calcolare il flusso del campo vettoriale
\end{enumerate}

\[
\mathbf{F}(\mathbf{x}, \mathbf{y}, \mathbf{z})=\left(y z^{2},-3 y^{2} z, x\right)
\]

attraverso la superficie

\[
S=\left\{(x, y, z) \in \mathbb{R}^{3}: y^{2}+z^{2} \leq 1, \quad x=3 y^{2}+z^{2}\right\}
\]

\begin{enumerate}
  \setcounter{enumi}{2}
  \item Calcolare
\end{enumerate}

\[
\iiint_{D}\left(x^{2}+y^{2}\right) \mathrm{d} x \mathrm{~d} y \mathrm{~d} z
\]

essendo

\[
D=\left\{(x, y, z) \in \mathbb{R}^{3}: \frac{x^{2}}{4}+y^{2}+z^{2} \leq 1, \quad x \leq 1\right\}
\]

\begin{enumerate}
  \setcounter{enumi}{3}
  \item Determinare gli estremi relativi della funzione definita dalla legge
\end{enumerate}

\[
f(x, y, z)=x^{2}+y^{2}+z^{2}-x .
\]

Trovare poi, se esistono, gli estremi assoluti nell'insieme

\[
D=\left\{(x, y, z) \in \mathbb{R}^{3}: x^{2}+\frac{y^{2}}{4}+\frac{z^{2}}{9} \leq 1\right\} .
\]

\begin{enumerate}
  \setcounter{enumi}{4}
  \item Data la successione di funzioni
\end{enumerate}

\[
\left\{\frac{(x-n)^{4}}{(x-n)^{4}+4}\right\}
\]

i) studiare la convergenza puntuale e uniforme in \(\mathbb{R}\);

ii) determinare gli intervalli in cui essa converge uniformemente.


\end{document}