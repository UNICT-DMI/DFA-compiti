\documentclass[10pt]{article}
\usepackage[utf8]{inputenc}
\usepackage[T1]{fontenc}
\usepackage{amsmath}
\usepackage{amsfonts}
\usepackage{amssymb}
\usepackage[version=4]{mhchem}
\usepackage{stmaryrd}
\usepackage{bbold}

\title{Università degli Studi di Catania 
 Corso di Laurea in Fisica 
 Prova scritta di Analisi Matematica 2 }

\author{}
\date{}


\begin{document}
\maketitle
30 settembre 2019

(1) Calcolare il flusso del campo vettoriale

\[
\mathbf{F}=\left(x, y, z^{2}\right)
\]

uscente dalla frontiera del dominio

\[
T=\left\{(x, y, z) \in \mathbb{R}^{3}:-1 \leq z \leq-x^{2}-y^{2}\right\}
\]

(2) Calcolare l'integrale curvilineo della forma differenziale

\[
\omega(x, y)=\left[1-\frac{y}{x}\right] e^{\frac{y}{x}} \mathrm{~d} x+e^{\frac{y}{x}} \mathrm{~d} y
\]

lungo la curva \(\varphi(t)=\left(2+\cos \left(\pi t^{2}\right), 1+t^{2}\right), t \in[0,1]\) orientata nel verso delle \(t\) crescenti.

(3) Data la funzione definita dalla legge

\[
f(x, y, z)=x^{2}+4 y^{2}+4 z^{2}+y z-2 x
\]

i) determinare gli estremi relativi in \(\mathbb{R}^{3}\);

ii) determinare gli estremi assoluti nell'insieme

\[
X=\left\{(x, y, z) \in \mathbb{R}^{3}: \quad x^{2}+y^{2}+z^{2} \leq 4, \quad x \geq 0\right\}
\]

(4) Calcolare

essendo

\[
\iint_{D} \frac{x^{2} y}{x^{2}+y^{2}} \mathrm{~d} x \mathrm{~d} y
\]

\[
D=\left\{(x, y) \in \mathbb{R}^{2}: \quad x^{2}+y^{2} \leq 9, \quad 0 \leq x \leq y\right\} .
\]


\end{document}