\documentclass[10pt]{article}
\usepackage[utf8]{inputenc}
\usepackage[T1]{fontenc}
\usepackage{amsmath}
\usepackage{amsfonts}
\usepackage{amssymb}
\usepackage[version=4]{mhchem}
\usepackage{stmaryrd}
\usepackage{bbold}

\title{Università degli Studi di Catania 
 Corso di Laurea in Fisica 
 Prova scritta di Analisi Matematica 2 }

\author{}
\date{}


\begin{document}
\maketitle
9 settembre 2019

(1) Calcolare il flusso del campo vettoriale

\[
\mathbf{F}=\left(x y z, y z, z^{2}\right)
\]

attraverso la superficie

\[
S=\left\{(x, y, z) \in \mathbb{R}^{3}: \quad z=x y, \quad x^{2}+y^{2} \leq 20, \quad y \geq x^{2}\right\}
\]

orientata con la normale verso l'alto.

(2) Per ogni \(n \in \mathbb{N}\) sia \(f_{n}\) la funzione definita dalla legge

\[
f_{n}(x)=n^{2} x^{3} e^{-n x^{2}} \quad \forall x \in \mathbb{R} \text {. }
\]

Studiare la convergenza puntuale della successione di funzioni \(\left\{f_{n}\right\}\) e determinare gli intervalli di \(\mathbb{R}\) in cui essa converge uniformemente.

(3) Data la funzione definita dalla legge

\[
f(x, y)=\left(x^{2}-4 y^{2}\right) e^{-x^{2}-y^{2}}
\]

i) determinare gli estremi relativi in \(\mathbb{R}^{2}\);

ii) determinare gli estremi assoluti nell'insieme

\[
X=\left\{(x, y) \in \mathbb{R}^{2}: \quad \frac{x^{2}}{4}+y^{2} \leq 1\right\} .
\]

(4) Calcolare

\[
\iiint_{D}|x y| z^{2} \mathrm{~d} x \mathrm{~d} y \mathrm{~d} z
\]

essendo

\[
D=\left\{(x, y, z) \in \mathbb{R}^{3}: \quad x^{2}+y^{2}+z^{2} \leq 1, \quad 0 \leq z \leq \sqrt{x^{2}+y^{2}}\right\}
\]


\end{document}