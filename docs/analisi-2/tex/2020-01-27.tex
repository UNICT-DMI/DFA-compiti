\documentclass[10pt]{article}
\usepackage[utf8]{inputenc}
\usepackage[T1]{fontenc}
\usepackage{amsmath}
\usepackage{amsfonts}
\usepackage{amssymb}
\usepackage[version=4]{mhchem}
\usepackage{stmaryrd}
\usepackage{bbold}

\begin{document}
Anno Accademico 2019-2020

Corso di Laurea in Fisica (L-30)

Prova scritta di Analisi Matematica 2 (9 CFU)

27 gennaio 2020

1 Sia \(f\) la funzione reale di due variabili reali definita dalla legge

\[
f(x, y)=(1-|x|) x^{2} y .
\]

(a) Studiare la differenziabilità di \(f\) nel suo insieme di definizione.

(b) Determinare, se esistono, gli estremi assoluti di \(f\) sul rettangolo \([-1,1] \times[0,1]\).

2 Calcolare il seguente integrale doppio:

\[
\iint_{T} x y \mathrm{~d} x \mathrm{~d} y
\]

dove

\[
T=\left\{(x, y) \in \mathbb{R}^{2}: x^{2}+(y-2)^{2} \leq 4, x^{2}+y^{2} \geq 4, x \geq 0\right\} .
\]

3 Calcolare il lavoro compiuto dal campo vettoriale

lungo la curva

\[
\mathbf{F}=\left(\frac{3 y^{2}}{9 y^{4}+x^{2}},-\frac{6 x y}{9 y^{4}+x^{2}}-2\right)
\]

\[
\varphi(t)=\left(\cos t, \sin ^{2} t\right), \quad t \in\left[0, \frac{\pi}{4}\right]
\]

orientata nel verso delle \(t\) crescenti.

4 Calcolare il flusso del campo vettoriale

\[
\mathbf{F}=\left(x y^{2}+z, x^{2} y+z^{2}, x y\right)
\]

uscente dal dominio

\[
D=\left\{(x, y, z) \in \mathbb{R}^{3}: \quad x^{2}+y^{2} \leq 1, x^{2}+y^{2}+z \geq 1, z \leq 3\right\} .
\]


\end{document}