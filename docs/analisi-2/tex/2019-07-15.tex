\documentclass[10pt]{article}
\usepackage[utf8]{inputenc}
\usepackage[T1]{fontenc}
\usepackage{amsmath}
\usepackage{amsfonts}
\usepackage{amssymb}
\usepackage[version=4]{mhchem}
\usepackage{stmaryrd}
\usepackage{bbold}

\title{Università degli Studi di Catania 
 Corso di Laurea in Fisica 
 Prova scritta di Analisi Matematica 2 }

\author{}
\date{}


\begin{document}
\maketitle
15 luglio 2019

(1) Calcolare il flusso del campo vettoriale

\[
\mathbf{F}=\left(x e^{-z}, z^{2} x, x y^{3}\right)
\]

attraverso il dominio

\[
D=\left\{(x, y, z) \in \mathbb{R}^{3}: \quad x^{2}+y^{2}+z \leq 2, \quad z \geq x^{2}+y^{2}\right\}
\]

(2) Data la forma differenziale

\[
\omega(x, y)=\frac{x-1}{(1-x)^{2}+y^{6}} \mathrm{~d} x+\frac{3 y^{5}}{(1-x)^{2}+y^{6}} \mathrm{~d} y
\]

calcolare

\[
\int_{\gamma} w
\]

essendo \(\gamma\) la curva che ha per sostegno l'arco di circonferenza \(x^{2}+y^{2}-2 x=0\) di estremi \((0,0)\) e \((2,0)\) percorso nel verso antiorario.

(3) Data la funzione funzione definita dalla legge

\[
f(x, y)=4 x^{2} y+y^{3}-4 y
\]

determinarne gli estremi relativi in \(\mathbb{R}^{2}\).

Determinarne poi gli estremi assoluti, se esistono, nel cerchio chiuso di centro l'origine e raggio 4 .

Stabilire, infine, se \(f\) è limitata in \(\mathbb{R}^{2}\).

(4) Studiare, al variare del parametro reale \(\alpha>0\) la continuità e la differenziabilità in \((0,0)\) della funzione definita dalla legge

\[
f(x, y)=\left\{\begin{array}{lll}
|x|^{\alpha} \frac{\sin y}{x^{2}+y^{2}} & \text { se } & (x, y) \neq(0,0) \\
0 & \text { se } & (x, y)=(0,0)
\end{array}\right.
\]


\end{document}