\documentclass[10pt]{article}
\usepackage[utf8]{inputenc}
\usepackage[T1]{fontenc}
\usepackage{amsmath}
\usepackage{amsfonts}
\usepackage{amssymb}
\usepackage[version=4]{mhchem}
\usepackage{stmaryrd}
\usepackage{bbold}

\title{Corso di Laurea in Fisica (L-30) }

\author{}
\date{}


\begin{document}
\maketitle
Anno Accademico 2020/21

Prova scritta di Analisi Matematica 2

24 gennaio 2022

1 Determinare gli eventuali estremi assoluti della funzione definita dalla legge

\[
f(x, y)=\arctan \sqrt{\frac{x^{2}-y^{2}}{x^{2}+y^{2}}}
\]

2 Calcolare il lavoro del campo vettoriale

\[
\mathbf{F}=\left(-y \frac{x^{2}+y^{2}}{\left(x^{2}-y^{2}\right)^{2}}, x \frac{x^{2}+y^{2}}{\left(x^{2}-y^{2}\right)^{2}}\right)
\]

lungo la curva che ha come sostegno l'arco di iperbole \(x^{2}-y^{2}=1\) di estremi \(A=(2, \sqrt{3})\) e \(B=(2,-\sqrt{3})\) percorsa nel verso che va da \(A\) a \(B\).

3 Calcolare il flusso del campo vettoriale

\[
\mathbf{F}=\left(2 x^{3} y,-3 x^{2} y^{2}, z^{2} x^{2}\right)
\]

uscente dal dominio

\[
X=\left\{(x, y, z) \in \mathbb{R}^{3}: z \geq \sqrt{x^{2}+y^{2}}, \quad x^{2}+y^{2}+z^{2} \leq 2\right\}
\]

4 Studiare la convergenza puntuale ed uniforme della serie

\[
\sum_{n=1}^{+\infty} \frac{\log n}{n^{2} \sqrt{4^{n}}}(x+1)^{2 n}
\]

Durata: 3 ore


\end{document}