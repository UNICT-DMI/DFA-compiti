\documentclass[10pt]{article}
\usepackage[utf8]{inputenc}
\usepackage[T1]{fontenc}
\usepackage{amsmath}
\usepackage{amsfonts}
\usepackage{amssymb}
\usepackage[version=4]{mhchem}
\usepackage{stmaryrd}
\usepackage{bbold}

\begin{document}
Anno Accademico 2020/21

Corso di Laurea in Fisica (L-30)

Prova scritta di Analisi Matematica 2

27 settembre 2021

1 Data la funzione definita dalla legge

\[
f(x, y)=\left(|x|-y^{2}\right) \exp (-x)
\]

i) studiare l'esistenza delle derivate parziali prime in tutti i punti del suo dominio;

ii) determinare gli eventuali punti di estremo relativo;

iii) stabilire, motivando la risposta, se la funzione è limitata.

2 Determinare \(a\) e \(b\) in modo che il campo vettoriale

\[
\mathbf{F}=\left(a x \log z, b y^{2} z, y^{3}+\frac{x^{2}}{z}\right)
\]

sia conservativo. Per tali valori di \(a\) e \(b\) calcolare il potenziale che si annulla nel punto \((1,1,1)\).

3 Calcolare il flusso del campo vettoriale

\[
\mathbf{F}=\left(x^{2}, x y, z\right)
\]

uscente dal tetraedro di vertici \((0,0,0),(2,0,0),(0,1,0)\) e \((0,0,1)\).

4 Calcolare il seguente integrale

\[
\iint_{X} \frac{x}{\sqrt{x^{2}+y^{2}}} y \mathrm{~d} x \mathrm{~d} y
\]

essendo

\[
X=\left\{(x, y) \in \mathbb{R}^{2}: x^{2}+y^{2} \leq 2 x, \quad x \geq \frac{3}{2}, \quad y \leq 0\right\}
\]

Durata: 3 ore


\end{document}