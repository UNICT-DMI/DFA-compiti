\documentclass[10pt]{article}
\usepackage[utf8]{inputenc}
\usepackage[T1]{fontenc}
\usepackage{amsmath}
\usepackage{amsfonts}
\usepackage{amssymb}
\usepackage[version=4]{mhchem}
\usepackage{stmaryrd}
\usepackage{bbold}

\title{Università degli Studi di Catania - Anno Accademico 2018/19 
 Corso di Laurea in Fisica 
 Prova scritta di Analisi Matematica 2 
 18 febbraio 2019 }

\author{}
\date{}


\begin{document}
\maketitle
\begin{enumerate}
  \item Determinare gli eventuali estremi relativi e assoluti della funzione definita dalla legge
\end{enumerate}

\[
f(x, y)=\frac{x y}{1+x^{4}+y^{4}}
\]

\begin{enumerate}
  \setcounter{enumi}{1}
  \item Calcolare
\end{enumerate}

\[
\int_{T} \frac{x-\sqrt{3} y}{\left(x^{2}+y^{2}\right)^{2}} d x d y
\]

essendo

\[
T=\left\{(x, y) \in \mathbb{R}^{2}: \quad x^{2}+y^{2}-2 y \leq 0, \quad \sqrt{3} x+y \geq 2\right\}
\]

\begin{enumerate}
  \setcounter{enumi}{2}
  \item Data la forma differenziale
\end{enumerate}

\[
\omega(x, y)=\left(x^{2}-y^{2}+f(y)\right) d x+x f(y) d y
\]

Determinare l'unica funzione \(f \in C^{1}(\mathbb{R})\) tale che \(\omega\) sia esatta in \(\mathbb{R}^{2} \mathrm{e}\) \(f(0)=0\). In corrispondenza di tale \(f\), determinare il potenziale che si annulla nell'origine.

\begin{enumerate}
  \setcounter{enumi}{3}
  \item Studiare la convergenza puntuale ed uniforme della serie
\end{enumerate}

\[
\sum_{n=1}^{+\infty} \frac{\log n}{n \sqrt{2^{n}}}(x-1)^{2 n} .
\]

\begin{enumerate}
  \setcounter{enumi}{4}
  \item Determinare il flusso del campo vettoriale
\end{enumerate}

\[
\mathbf{F}(x, y, z)=\left(x^{2}, y^{2}, z^{2}\right)
\]

uscente dal tetraedro di vertici \((0,0,0),(1,0,0),(0,1,0)\) e \((0,0,1)\).


\end{document}